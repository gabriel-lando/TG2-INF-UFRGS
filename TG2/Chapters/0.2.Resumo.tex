O presente trabalho desenvolve uma prova de conceito de um módulo para execução de testes de desempenho em implementações de núcleos de redes 5G.
O objetivo deste trabalho é analisar como se comportam as diferentes implementações de código aberto dos núcleos de rede 5G, \textit{free5GC}, \textit{Open5GS} e \textit{OpenAirInterface}, para a execução de procedimentos em escala.
Para responder a esse problema, primeiramente foi feito um um breve resumo sobre a evolução das redes móveis, com foco em implementações em software de núcleos de redes 5G.
A seguir, são apresentados os trabalhos relacionados, assim como as diferenças entre eles e a presente pesquisa.
Após, uma arquitetura de software foi desenvolvida, criando um módulo de extensão para o testador \textit{my5G-RANTester} para realizar testes de desempenho nessas implementações de código aberto de núcleos de redes 5G.
Foram executados experimentos sobre essas implementações.
Na primeira fase dos experimentos, foi feito um teste para analisar o tempo médio de conexão de cada equipamento de usuário com cada núcleo testado. Na segunda etapa do experimento, foi medida a largura de banda do plano de dados entre o equipamento de usuário e o núcleo da rede.
Dentre os principais resultados obtidos, foi possível observar que o \textit{free5GC} apresenta melhor desempenho em relação à largura de banda disponível para os equipamentos de usuário, enquanto que o \textit{Open5GS} apresenta mais estabilidade durante o processo de registro de múltiplos equipamentos de usuário.
Este trabalho possui duas importantes contribuições teóricas para a literatura. A primeira contribuição é a agregação de conhecimento sobre testes de desempenho em redes móveis, enquanto a segunda é relacionada com a estabilidade e limitações das implementações de núcleos 5G em software.
A principal contribuição prática deste trabalho é o desenvolvimento de um módulo para realizar testes de desempenho em núcleos de rede 5G.
Como sugestões de futuras pesquisas, é possível avançar a investigação sobre experimentos em núcleos comerciais ou replicar os testes em ambientes de fácil escalabilidade.

