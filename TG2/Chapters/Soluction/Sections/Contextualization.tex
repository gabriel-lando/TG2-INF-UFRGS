Este trabalho faz parte do projeto de pesquisa \textit{PORVIR-5G}\footnote{https://porvir-5g-project.github.io/}, um projeto colaborativo entre diversas universidades brasileiras com objetivo de conceber uma arquitetura aberta que ofereça programabilidade e orquestração para fatiamento de rede, com consistência semântica na expressividade.
O projeto de desenvolvimento do testador \textit{my5G-RANTester}\footnote{https://github.com/my5G/my5G-RANTester} tem relação com o projeto de pesquisa \textit{PORVIR-5G} ao demonstrar a viabilidade da arquitetura PORVIR-5G em diversos casos de uso explorando requisitos avançados das redes 5G, especialmente em termos de latência, confiabilidade, cobertura, mobilidade e banda.
A presente pesquisa teve a contribuição dos bolsistas de iniciação científica Lucas Schierholt\footnote{https://github.com/lucas-schierholt} e Mateus Milesi \textcolor{red}{Adicionar alguma URL} no desenvolvimento do módulo de coleta de dados do núcleo.
