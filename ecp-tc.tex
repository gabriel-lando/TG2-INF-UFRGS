%Template para TG2 ECP UFRGS
%Author: Lucas Schnorr (Adaptado para overleaf por Rubens dos Santos)
% exemplo genérico de uso da classe iiufrgs.cls
% $Id: iiufrgs.tex,v 1.1.1.1 2005/01/18 23:54:42 avila Exp $
%
% This is an example file and is hereby explicitly put in the
% public domain.
%
\documentclass[ecp,tc]{iiufrgs}
% Para usar o modelo, deve-se informar o programa e o tipo de documento.
% Programas :
%   * cic       -- Graduação em Ciência da Computação
%   * ecp       -- Graduação em Ciência da Computação
%   * ppgc      -- Programa de Pós Graduação em Computação
%   * pgmigro   -- Programa de Pós Graduação em Microeletrônica
%   
% Tipos de Documento:
%   * tc                -- Trabalhos de Conclusão (apenas cic e ecp)
%   * diss ou mestrado  -- Dissertações de Mestrado (ppgc e pgmicro)
%   * tese ou doutorado -- Teses de Doutorado (ppgc e pgmicro)
%   * ti                -- Trabalho Individual (ppgc e pgmicro)
% 
% Outras Opções:
%   * english    -- para textos em inglês
%   * openright  -- Força início de capítulos em páginas ímpares (padrão da
%                   biblioteca)
%   * oneside    -- Desliga frente-e-verso
%   * nominatalocal -- Lê os dados da nominata do arquivo nominatalocal.def


% Use unicode
\usepackage[utf8]{inputenc}   % pacote para acentuação

% Necessário para incluir figuras
\usepackage{graphicx}           % pacote para importar figuras


\usepackage{times}              % pacote para usar fonte Adobe Times
% \usepackage{palatino}
% \usepackage{mathptmx}          % p/ usar fonte Adobe Times nas fórmulas

\usepackage[alf,abnt-emphasize=bf]{abntex2cite}	% pacote para usar citações abnt

%
% Informações gerais
%
\title{Um Exemplo de Monografia do Instituto de Informática da UFRGS}

\author{Flaumann}{Fritz Gutenberg}
% alguns documentos podem ter varios autores:
%\author{Flaumann}{Frida Gutenberg}
%\author{Flaumann}{Klaus Gutenberg}

% orientador e co-orientador são opcionais (não diga isso pra eles :))
\advisor[Prof.~Dr.]{Lamport}{Leslie}
%\coadvisor[Prof.~Dr.]{Knuth}{Donald Ervin}

% a data deve ser a da defesa; se nao especificada, são gerados
% mes e ano correntes
%\date{maio}{2001}

% o local de realização do trabalho pode ser especificado (ex. para TCs)
% com o comando \location:
%\location{Itaquaquecetuba}{SP}

% itens individuais da nominata podem ser redefinidos com os comandos
% abaixo:
% \renewcommand{\nominataReit}{Prof\textsuperscript{a}.~Wrana Maria Panizzi}
% \renewcommand{\nominataReitname}{Reitora}
% \renewcommand{\nominataPRE}{Prof.~Jos{\'e} Carlos Ferraz Hennemann}
% \renewcommand{\nominataPREname}{Pr{\'o}-Reitor de Ensino}
% \renewcommand{\nominataPRAPG}{Prof\textsuperscript{a}.~Joc{\'e}lia Grazia}
% \renewcommand{\nominataPRAPGname}{Pr{\'o}-Reitora Adjunta de P{\'o}s-Gradua{\c{c}}{\~a}o}
% \renewcommand{\nominataDir}{Prof.~Philippe Olivier Alexandre Navaux}
% \renewcommand{\nominataDirname}{Diretor do Instituto de Inform{\'a}tica}
% \renewcommand{\nominataCoord}{Prof.~Carlos Alberto Heuser}
% \renewcommand{\nominataCoordname}{Coordenador do PPGC}
% \renewcommand{\nominataBibchefe}{Beatriz Regina Bastos Haro}
% \renewcommand{\nominataBibchefename}{Bibliotec{\'a}ria-chefe do Instituto de Inform{\'a}tica}
% \renewcommand{\nominataChefeINA}{Prof.~Jos{\'e} Valdeni de Lima}
% \renewcommand{\nominataChefeINAname}{Chefe do \deptINA}
% \renewcommand{\nominataChefeINT}{Prof.~Leila Ribeiro}
% \renewcommand{\nominataChefeINTname}{Chefe do \deptINT}

% A seguir são apresentados comandos específicos para alguns
% tipos de documentos.

% Relatório de Pesquisa [rp]:
% \rp{123}             % numero do rp
% \financ{CNPq, CAPES} % orgaos financiadores

% Trabalho Individual [ti]:
% \ti{123}     % numero do TI
% \ti[II]{456} % no caso de ser o segundo TI

% Monografias de Especialização [espec]:
% \espec{Redes e Sistemas Distribuídos}      % nome do curso
% \coord[Profa.~Dra.]{Weber}{Taisy da Silva} % coordenador do curso
% \dept{INA}                                 % departamento relacionado

%
% palavras-chave
% iniciar todas com letras minúsculas, exceto no caso de abreviaturas
%
\keyword{formatação eletrônica de documentos}
\keyword{\LaTeX}
\keyword{ABNT}
\keyword{UFRGS}

%
% inicio do documento
%
\begin{document}

% folha de rosto
% às vezes é necessário redefinir algum comando logo antes de produzir
% a folha de rosto:
% \renewcommand{\coordname}{Coordenadora do Curso}
\maketitle

% dedicatoria
\clearpage
\begin{flushright}
\mbox{}\vfill
{\sffamily\itshape
``If I have seen farther than others,\\
it is because I stood on the shoulders of giants.''\\}
--- \textsc{Sir~Isaac Newton}
\end{flushright}

% agradecimentos
\chapter*{Agradecimentos}
Agradeço ao \LaTeX\ por não ter vírus de macro\ldots



% resumo na língua do documento
\begin{abstract}
Este documento é um exemplo de como formatar documentos para o
Instituto de Informática da UFRGS usando as classes \LaTeX\
disponibilizadas pelo UTUG\@. Ao mesmo tempo, pode servir de consulta
para comandos mais genéricos. \emph{O texto do resumo não deve
conter mais do que 500 palavras.}
\end{abstract}

% resumo na outra língua
% como parametros devem ser passados o titulo e as palavras-chave
% na outra língua, separadas por vírgulas
\begin{englishabstract}{Using \LaTeX\ to Prepare Documents at II/UFRGS}{Electronic document preparation, \LaTeX, ABNT, UFRGS}
This document is an example on how to prepare documents at II/UFRGS
using the \LaTeX\ classes provided by the UTUG\@. At the same time, it
may serve as a guide for general-purpose commands. \emph{The text in
the abstract should not contain more than 500~words.}
\end{englishabstract}

% lista de abreviaturas e siglas
% o parametro deve ser a abreviatura mais longa
\begin{listofabbrv}{SPMD}
        \item[SMP] Symmetric Multi-Processor
        \item[NUMA] Non-Uniform Memory Access
        \item[SIMD] Single Instruction Multiple Data
        \item[SPMD] Single Program Multiple Data
        \item[ABNT] Associação Brasileira de Normas Técnicas
\end{listofabbrv}

% idem para a lista de símbolos
%\begin{listofsymbols}{$\alpha\beta\pi\omega$}
%       \item[$\sum{\frac{a}{b}}$] Somatório do produtório
%       \item[$\alpha\beta\pi\omega$] Fator de inconstância do resultado
%\end{listofsymbols}

% lista de figuras
\listoffigures

% lista de tabelas
\listoftables

% sumario
\tableofcontents

% aqui comeca o texto propriamente dito

% introducao
\chapter{Introdução}
No início dos tempos, Donald E. Knuth criou o \TeX. Algum tempo depois, Leslie Lamport criou o \LaTeX. Graças a eles, não somos obrigados a usar o Word nem o LibreOffice.

\section{Figuras e tabelas}

Esta seção faz referência às Figuras~\ref{fig:estrutura},~\ref{fig:ex2}, a título de exemplo. A primeira figura apresenta a estrutura de uma figura. A \emph{descrição} deve aparecer \textbf{acima} da figura. Abaixo da figura, deve ser indicado a origem da imagem, mesmo se essa for apenas os autores do texto.

 A Figura~\ref{fig:ex2} exemplifica o uso do environment \texttt{picture}, para desenhar usando o próprio~\LaTeX.

\begin{figure}[h]
    \caption{Descrição da Figura deve ir no topo}
    \begin{center}
        % Aqui vai um includegraphics , um picture environment ou qualquer
        % outro comando necessário para incorporar o formato de imagem
        % utilizado.
        \begin{picture}(100,100)
                \put(0,0){\line(0,1){100}}
                \put(0,0){\line(1,0){100}}
                \put(100,100){\line(0,-1){100}}
                \put(100,100){\line(-1,0){100}}
                \put(10,50){Uma Imagem}
        \end{picture}
    \end{center}
    \label{fig:estrutura}
    \legend{Fonte: Os Autores}
\end{figure}

% o `[h]' abaixo é um parâmetro opcional que sugere que o LaTeX coloque a
% figura exatamente neste ponto do texto. Somente preocupe-se com esse tipo
% de formatação quando o texto estiver completamente pronto (uma frase a mais
% pode fazer o LaTeX mudar completamente de idéia sobre onde colocar as
% figuras e tabelas)
%\begin{figure}[h]
\begin{figure}
    \caption{Exemplo de figura desenhada com o environment \texttt{picture}.}
    \begin{center}
        \setlength{\unitlength}{.1em}
        \begin{picture}(100,100)
                \put(20,20){\circle{20}}
                \put(20,20){\small\makebox(0,0){a}}
                \put(80,80){\circle{20}}
                \put(80,80){\small\makebox(0,0){b}}
                \put(28,28){\vector(1,1){44}}
        \end{picture}
    \end{center}
    \legend{Fonte: Os Autores}
    \label{fig:ex2}
\end{figure}

Tabelas são construídas com praticamente os mesmos comandos. Ver a tabela \ref{tbl:ex1}.

\begin{table}[h]
    \caption{Uma tabela de Exemplo}
    \begin{center}
        \begin{tabular}{c|c|p{5cm}}
            \textit{Col 1}  &   \textit{Col 2}  &   \textit{Col 3} \\
            \hline
            \hline
            Val 1           &   Val 2           & Esta coluna funciona como um parágrafo, tendo uma margem definida em 5cm. Quebras de linha funcionam como em qualquer parágrafo do tex. \\
            Valor Longo     & Val 2             & Val 3 \\
            \hline
        \end{tabular}
    \end{center}
    \legend{Fonte: Os Autores}
    \label{tbl:ex1}
\end{table}

\subsection{Formato de Figuras}
\label{sec:fig_format}

O LaTeX permite utilizar vários formatos de figuras, entre eles \emph{eps}, \emph{pdf}, \emph{jpeg} e \emph{png}. Programas de diagramação como Inkscape (e mesmo LibreOffice) permitem gerar arquivos de imagens vetoriais que podem ser utilizados pelo LaTeX sem dificuldade. Pacotes externos permitem utilizar SVG e outros formatos.

Dia e xfig são programas utilizados por dinossauros para gerar figuras vetoriais. Se possível, evite-os.

\subsection{Classificação dos etc.}

O formato adotado pela ABNT prevê apenas três níveis (capítulo, seção e subseção). Assim, \texttt{\char'134subsubsection} não é aconselhado.

\section{Sobre as referências bibliográficas}

A classe \emph{iiufrgs} faz uso do pacote \emph{abnTeX2} com algumas alterações
feitas por Sandro Rama Fiorini. Culpe ele se algo der errado. Agradeça a ele
pelo que der certo. As modificações dão uma camada de tinta NATBIB-style,
já que o abntex2 usa uns comandos de citação feitos para alienígenas de 5 braços 
wtf. Exemplos de citação:

\begin{itemize}
    \item \emph{cite}: Unicórnios são verdes \cite{Adams2009Conceptual};
    \item \emph{citep}:Unicórnios são verdes \citep{Adams2009Conceptual};
    \item \emph{citet}: Segundo \citet{Adams2009Conceptual}, unicórnios são
                        verdes.
    \item \emph{citen or citenum}: Segundo \citen{Adams2009Conceptual},
        unicórnios são verdes.
    \item \emph{citeauthor e citeyearpar}: Segundo artigos de
        \citeauthor{Adams2009Conceptual} , unicórnios são verdes 
        \citeyearpar{Adams2009Conceptual}.

\end{itemize}

O estilo abnt fornecido antigamente pelo UTUG não é mais recomendado, pois não
produz saída de acordo com as exigências da biblioteca.

Recomenda-se o uso de bibtex para gerenciar as referências (veja o arquivo
biblio.bib).

% e aqui vai a parte principal
%
% \chapter{Estado da arte}
% \chapter{Mais estado da arte}
% \chapter{A minha contribuição}
% \chapter{Prova de que a minha contribuição é válida}
% \chapter{Conclusão}

% referencias
% aqui será usado o environment padrao `thebibliography'; porém, sugere-se
% seriamente o uso de BibTeX e do estilo abnt.bst (veja na página do
% UTUG)
%
% observe também o estilo meio estranho de alguns labels; isso é
% devido ao uso do pacote `natbib', que permite fazer citações de
% autores, ano, e diversas combinações desses

\bibliographystyle{abntex2-alf}
\bibliography{biblio}

\end{document}
