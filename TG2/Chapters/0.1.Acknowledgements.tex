Aos meus pais, Ernesto Antônio Lando e Izidora Justina Bizotto Lando, e à minha irmã, Anelise Lando, que me apoiaram, acreditaram em mim e me deram suporte desde que saí de casa para vir morar sozinho em outro Estado para cursar Engenharia de Computação em uma das melhores universidades do país. Muito obrigado.

À Isadora, minha parceira de vida, que esteve ao meu lado nesses últimos anos, me apoiando, incentivando e ouvindo eu reclamar de tudo.
Muito obrigado, Isa, por toda essa parceria, conselhos e ajuda que você me deu. Eu evoluí muito como pessoa ao seu lado.

Ao Felipe Lando, meu irmão, que me acolhe desde as tentativas falhas do vestibular da UFRGS até hoje, juntamente com a minha cunhada, Amália Machado.
Muito obrigado por todos esses anos de parceria, muitos almoços, jantas e projetos que fizemos juntos desde que eu vim morar na Capital.
À Amália, um agradecimento especial por toda a ajuda que você me deu durante o desenvolvimento deste trabalho.
Se eu estou aqui hoje é principalmente graças a vocês dois e a Acadêmica Pesquisa.

Ao Urso, meu cachorro, que esteve ao meu lado durante todo o desenvolvimento deste trabalho, seja me mordendo, latindo, pulando por cima de mim querendo brincar ou só deitado em silêncio me fazendo companhia.
Eu sei que você não entende a importância do que eu estava fazendo, mas muito obrigado pelo carinho.

Aos meus colegas de trabalho da HP, que compreenderam as diversas vezes que precisei me ausentar para poder realizar atividades referentes a este trabalho.
Aqui vai um agradecimento especial à Thayná Minuzzo, que além de colega de trabalho, também foi minha colega durante a graduação, me ajudando em diversas disciplinas, incluindo dicas durante a realização do presente trabalho.
Vocês também tem uma parte desta minha conquista. Muito obrigado.

Aos meus colegas e amigos que estão comigo desde a primeira semana de aula, como a Ana e a Narumi, o Fischer, o Probst e o Rodrigo, além de outras amizades que eu fiz durante todos esses anos de graduação.
Vocês me confortaram, muitas vezes indiretamente, só de saber que todos nós estávamos passando pelas mesmas dificuldades durante toda a graduação, incluindo nesse trabalho.

Aos meus padrinhos e primos, Tia Enedina, Tio Ronei, Leonardo, Ramon, Lucas e Marina, que sempre me incentivaram e me acolheram nos feriados e finais de semana.
Muito obrigado por esse tempo juntos, me ajudaram a esfriar a cabeça nos momentos mais difíceis da graduação.

Ao meu orientador, Prof. Dr. Juliano Wickboldt, que me ofereceu a primeira oportunidade de bolsa de pesquisa, lá em 2017, no meu segundo semestre de graduação, com o Projeto FUTEBOL e, em 2021, aceitou ser meu orientador para o desenvolvimento desta pesquisa.
Aqui também vão meus agradecimentos aos bolsistas de iniciação científica do projeto PORVIR-5G, Lucas Schierholt e Mateus Milesi, que me ajudaram no desenvolvimento deste trabalho.
Sem vocês, eu não teria chegado até aqui.

Por fim, agradeço a todos aqueles que, de alguma forma, contribuíram para que este trabalho fosse concluído.
Espero que os resultados obtidos ajudem a desenvolver essa área da computação.
