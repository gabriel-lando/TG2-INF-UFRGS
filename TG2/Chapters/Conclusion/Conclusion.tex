O presente trabalho atinge a proposta inicial, desenvolvendo uma ferramenta para testes de desempenho de redes móveis 5G de código aberto, além de avaliar o desempenho em dois distintos cenários com duas implementações de núcleo 5G existentes.
As descobertas e limitações encontradas podem ser úteis para o desenvolvimento dos núcleos. Da mesma forma, essa ferramenta pode ser utilizada para avaliar outros núcleos não testados no presente trabalho.

Este trabalho apresenta duas principais contribuições teóricas. A primeira é com relação aos testes de desempenho, principalmente sobre redes móveis.
A implementação do módulo de execução dos testes de desempenho e análise dos dados coletado traz discussões relevantes sobre a utilidade do uso de testes de desempenho durante o desenvolvimento de software.
Essas discussões são de extrema relevância para a área de Engenharia de Software, mostrando que os testes de desempenho ajudam a validar implementações de software para operarem de maneira correta sobre alta carga de trabalho.

A segunda contribuição teórica deste trabalho é em relação à estabilidade das atuais implementações de código aberto de núcleos de rede 5G.
As três implementações avaliadas possuem grandes limitações em relação ao desempenho em escala, tornando sua implantação em redes reais inviável neste momento.
Esse trabalho demonstra que a implementação OAI, sendo a menos madura delas, não conseguiu rodar nenhum dos testes de desempenho devido a suas limitações.
Por outro lado, as outras duas implementações avaliadas apresentam diversas limitações que fizeram com que o módulo de testes de desempenho desenvolvido fosse adaptado para funcionar sobre as limitações encontradas e descritas no decorrer do trabalho.

Este trabalho também possui contribuições práticas, ao desenvolver um módulo para ser acoplado no testador \textit{my5G-RANTester} de núcleos 5G.
Este módulo está disponível no repositório de código do projeto PORVIR-5G e é aberto para a comunidade.
Dessa forma, é possível utilizar a implementação do testador disponível para realizar testes sobre novas versões das implementações existentes de núcleos 5G e para outras implementações futuras.

Este trabalho possui algumas limitações.
Dentre elas, pode-se citar as limitações dos núcleos testados, fazendo com que os testes em escala fossem limitados a uma menor quantidade de UEs se registrando, estabelecendo uma sessão PDU e trafegando dados.
Outra limitação relevante é em relação à execução do testador na mesma máquina virtual onde o núcleo foi executado.
Ao executar todo o ambiente de testes na mesma máquina que está sendo executado o núcleo da rede 5G, os recursos de hardware disponíveis são divididos entre os processos em execução.
Dessa forma, o uso de memória RAM e processador pelo testador para gerenciar as diversas conexões de UEs e pelos UEs para executar testes de largura de banda reduz a quantidade de recursos disponíveis para o núcleo, afetando o seu desempenho.

O presente trabalho abre oportunidades para futuros experimentos em relação ao desempenho de núcleos de redes 5G.
Um exemplo seria o teste sobre outras implementações de núcleos de redes 5G de código aberto não avaliadas no presente trabalho.
Outro tópico para pesquisas seria avaliar o desempenho de implementações comerciais de núcleos de redes 5G.
Além disso, pode-se utilizar a implementação atual do testador, executando testes sobre ambientes de fácil escalabilidade, permitindo executar o testador em uma máquina virtual separada da máquina que está executando o núcleo a ser testado.
Dessa forma, o uso de recursos do testador não deve afetar a disponibilidade de recursos para o núcleo, tendo resultados mais relevantes em relação ao funcionamento da implementação em teste.
