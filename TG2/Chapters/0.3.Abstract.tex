With the advent of 5G networks, open source implementations of their cores appear, which must follow a specification described by 3GPP.
To validate the implementation, it is necessary to perform tests on these 5G network cores.
The present work develops a proof of concept of a module to run performance tests in open source implementations of 5G network cores.
The objective of this work is to analyze how the different open source implementations of the 5G network cores free5GC, Open5GS and OpenAirInterface behave, for the execution of procedures at scale.
To answer this problem, a brief summary of the evolution of mobile networks was first made, focusing on software implementations of 5G network cores.
Afterwards, a software architecture was developed, creating an extension module for the my5G-RANTester tester to run performance tests on these open source implementations of 5G network cores.
Two different experiments were performed on these implementations, being them measuring the registration time of each user equipment and measuring the data plane bandwidth between the user equipment and the network core.
Among the main results obtained, it was possible to observe that free5GC presents better performance in relation to the bandwidth available to user devices, while Open5GS presents more stability during the registration process of multiple user devices.
This work has two important theoretical contributions to the literature. The first contribution is the aggregation of knowledge about performance testing in mobile networks, while the second is related to the stability and limitations of 5G core implementations in software.
The main practical contribution of this work is the development of a module to perform performance tests on 5G network cores.