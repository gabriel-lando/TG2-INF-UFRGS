Este trabalho faz parte do projeto de pesquisa \textit{Programmability, Orchestration and Virtualization on 5G networks} (PORVIR-5G)\footnote{https://porvir-5g-project.github.io/}, um projeto colaborativo entre diversas universidades brasileiras. O principal objetivo desse projeto é conceber uma arquitetura aberta que ofereça programabilidade e orquestração para fatiamento de rede.
O projeto de desenvolvimento do testador \textit{my5G-RANTester} tem relação com o projeto de pesquisa PORVIR-5G ao demonstrar a viabilidade da arquitetura desenvolvida em diversos casos de uso explorando requisitos avançados das redes 5G, especialmente em termos de latência, confiabilidade, cobertura, mobilidade e banda.
A presente pesquisa teve a contribuição de dois bolsistas de iniciação científica do projeto PORVIR-5G.
